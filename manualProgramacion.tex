%Basado en los manuales de Loperamida Clorhidrato 2mg y Andr�s Mej�a
\documentclass[10pt,letterpaper,twocolumn,twosided]{article}

\usepackage[utf8]{inputenc}
\usepackage[spanish]{babel}
\usepackage{listings}
\usepackage[usenames,dvipsnames]{color}
\usepackage{amsmath}
\usepackage{verbatim}
\usepackage{hyperref}
\usepackage{color}
\usepackage{geometry}
\usepackage{amssymb}

\geometry{verbose,landscape,letterpaper,tmargin=2cm,bmargin=2cm,lmargin=2cm,rmargin=2cm}
% ---------------------------------------------------------------
% ---------------------------------------------------------------

% ---------------------------------------------------------------
%Comando \codigofuente{} para importar directamente el archivo .cpp o .java
% ---------------------------------------------------------------
\newcommand{\codigofuente}[1]{
	\verbatiminput{#1}
	\dotfill
}
\setlength{\columnsep}{0.25in}
\setlength{\columnseprule}{1px}
% ---------------------------------------------------------------
% ---------------------------------------------------------------

% ---------------------------------------------------------------
% ---------------------------------------------------------------
\begin{document}
\title{Resumen de algoritmos para competencias de programaci\'on}
\author{Kruskal}

\maketitle
% ---------------------------------------------------------------
% ---------------------------------------------------------------

% ---------------------------------------------------------------
% ---------------------------------------------------------------
%Para quitar el �nidice.... comentar la l�nea de abajo

%\lstlistoflistings
\tableofcontents
%\lstloadlanguages{C++}

% ---------------------------------------------------------------
% ---------------------------------------------------------------

% ---------------------------------------------------------------
% ---------------------------------------------------------------

\section{Plantillas}

\subsection{C++}
\codigofuente{./src/template.cpp}

\subsection{Entradas especiales con C++}
\codigofuente{./src/inputs.cpp}

\subsection{Java}
\codigofuente{./src/template.java}

% ---------------------------------------------------------------
% ---------------------------------------------------------------

% ---------------------------------------------------------------
% ---------------------------------------------------------------

\section{Matem\'atica Aplicada}
% ---------------------------------------------------------------

% ---------------------------------------------------------------

\subsection{Suma de enteros, suma de cuadrados y suma de cubos}

	\begin{equation*}
		\sum_{i = 1}^{n} i = 1 + 2 +... + n = \frac{n(n+1)}{2}
	\end{equation*}
	
	\begin{equation*}
		\sum_{i = 1}^{n} i^2 = 1^2 + 2^2 +... + n^2 = \frac{(2n+1)n(n+1)}{6}
	\end{equation*}
	
	\begin{equation*}
		\sum_{i = 1}^{n} i^3 = 1^3 + 2^3 +... + n^3 = \frac{n^2(n+1)^2}{4}
	\end{equation*}
% ---------------------------------------------------------------

% ---------------------------------------------------------------

\subsection{Conjunto de partes}

De un conjunto $S$, el \textbf{conjunto de partes} P($S$) son todos los posibles subconjuntos que se pueden formar con los elementos de $S$. Si el cardinal de $S$ es de $n$, el cardinal de su conjunto de partes P($S$) ser\'a de $2^n$ e incluye el conjunto vac\'io ($ \emptyset$).
\\
\\
\textbf{Field-testing:}
\begin{itemize}
\item \emph{UVa Live Archive} - 4794 - Sharing Chocolate
\item \emph{SPOJ} - 6073 - Chocolate
\end{itemize}
\codigofuente{./src/bitmask.cpp}
% ---------------------------------------------------------------

% ---------------------------------------------------------------
\subsection{Fibonacci $O(log \ n)$}
\codigofuente{./src/fibonacci.cpp}
% ---------------------------------------------------------------

% ---------------------------------------------------------------

\subsection{Geometr\'ia}

\begin{center}
	\'Area de un pol\'igono. (Si no se intersecta a s\'i mismo, es simple): 
	\begin{equation*}
		A(P) = \frac{1}{2} \sum_{i = 0}^{n-1}(x_i \cdot y_{i+1} - x_{i+1} \cdot y_i)
	\end{equation*}
\end{center}
\begin{center}
Centro de masa de un pol\'igono simple con \'area M: 
\begin{equation*}
	\bar{C}_x= \frac{\iint \limits_{R}xdA}{M} = \frac{1}{6M} \sum_{i = 1}^{n}(y_{i+1} \cdot y_i)(x_{i+1}^2 + x_{i+1} \cdot x_i + x_i^2)
\end{equation*}
\begin{equation*}
	\bar{C}_y= \frac{\iint \limits_{R}ydA}{M} = \frac{1}{6M} \sum_{i = 1}^{n}(x_{i+1} \cdot x_i)(y_{i+1}^2 + y_{i+1} \cdot y_i + y_i^2)
\end{equation*}
\end{center}
% ---------------------------------------------------------------

% ---------------------------------------------------------------
\subsection{\'Areas y vol\'umenes}

%Para dos dimensiones------>
Para $\mathbb{R}^2$ con tres puntos $P(x_1,y_1), R(x_2,y_2),Q(x_3,y_3)$ NO colineales se tiene que el \'area del tri\'angulo es:
\begin{equation*}
	\acute{A}rea \vartriangle \ = \frac{1}{2} \left | det \begin{bmatrix}
	x_1 & y_1 & 1 \\ 
	x_2 & y_2 & 1\\ 
	x_3 & y_3 & 1
	\end{bmatrix} \right |  
\end{equation*}
\begin{equation*}
	\acute{A}rea \vartriangle \  = \frac{(x_2y_1 - x_1y_2 + x_3y_2 - x_2y_3 - x_3y_1 +x_1y_3)}{2}
\end{equation*}
\\
\\
%Para tres dimensiones------>
Teniendo tres puntos $P,Q,R$ NO colineales en $\mathbb{R}^3$ el \'area del tri\'angulo que forman es:
\begin{equation*}
	\acute{A}rea \vartriangle \ = \frac{\| \overrightarrow{PR} \times \overrightarrow{PQ} \|}{2}
\end{equation*}
Como $\acute{A}rea_{paralelogramo} = 2\cdot \acute{A}rea\vartriangle \longrightarrow \acute{A}rea_{paralelogramo}  = \| \overrightarrow{PR} \times \overrightarrow{PQ} \|$
\\ \\
%Volumen paralelepipedo
Para encontrar el volumen de un paralelep\'ipedo de manera vectorial, se deben conocer $\vec{u}, \vec{v}, \vec{w}$ y aplicar la f\'ormula: $V_p = (\Vec{u} \times \Vec{v}) \cdot \Vec{w}$
% ---------------------------------------------------------------

% ---------------------------------------------------------------
\subsection{Algoritmo de Josefo}

Se trata de encontrar el lugar en el c\'irculo inicial para que se pueda ser el \'ultimo y seguir viviendo. (Basado en el problema que enfrent\'o Josefo).
\\ \\
\textbf{Field-testing:}
\begin{itemize}
\item \emph{SPOJ} - 4557 - Musical Chairs
\end{itemize}
\codigofuente{./src/josephus.cpp}

% ---------------------------------------------------------------

% ---------------------------------------------------------------

\subsection{Particiones de $A$ condicionadas}

\begin{equation}
	\forall X \{X \in P  | X \neq \emptyset \}
\end{equation}

\begin{equation}
	\forall (X,Y) \{X \in P, Y \in P | X \neq Y \longrightarrow X \cap Y = \emptyset \}
\end{equation}

\begin{equation}
	  \{ \bigcup_{i = 1}^n X_i \ | \ X_i \in P \}  = A 
\end{equation}

entonces: \\ \\
\begin{equation}
	S(n,k) = S( n-1, k-1 ) + k \times S( n-1, k ) 
\end{equation}

\dotfill
% ---------------------------------------------------------------
% ---------------------------------------------------------------

% ---------------------------------------------------------------
% ---------------------------------------------------------------
\section{Combinatoria}

\subsection{Cuadro resumen}

F\'ormulas para combinaciones y permutaciones:
\begin{center}
  \renewcommand{\arraystretch}{2} %Multiplica la altura de cada fila de la tabla por 2
  % Si quiero aumentar el tama�o de una fila en particular insertar \rule{0cm}{1cm} en esa fila.
  \begin{tabular}{| c | c | c |}
    \hline
    \textit{Tipo} & \textit{\textquestiondown Repetici\'on permitida?} & \textit{F\'ormula} \\ [1.5ex]
    \hline\hline

    $r$-permutaciones & No & $ \displaystyle\frac{n!}{(n-r)!} $ \\ [1.5ex]
    \hline
    $r$-combinaciones & No & $ \displaystyle\frac{n!}{r!(n-r)!} $ \\ [1.5ex]
    \hline
    $r$-permutaciones & S\'i & $ \displaystyle n^{r} $ \\
    \hline
    $r$-combinaciones & S\'i & $ \displaystyle\frac{(n+r-1)!}{r!(n-1)!} $ \\ [1.5ex]
    \hline
  \end{tabular}
  \renewcommand{\arraystretch}{1}
\end{center}
Tomado de \textit{Matem\'atica discreta y sus aplicaciones}, Kenneth Rosen, 5${}^{\hbox{ta}}$ edici\'on, McGraw-Hill, p\'agina 315.
% ---------------------------------------------------------------

% ---------------------------------------------------------------

\subsection{Criba de Erat\'ostenes}

\small
\textbf{Field-testing:}
\begin{itemize}
\item \emph{SPOJ} - 2 - Prime Generator

\end{itemize}
\codigofuente{./src/prime-generator.cpp}
% ---------------------------------------------------------------

% ---------------------------------------------------------------
\subsection{Combinaciones, coeficientes binomiales, tri\'angulo de Pascal}

\emph{Complejidad:} $ O(n^2) $ \\
$$ {n \choose k} = \left\{
\begin{array}{c l}
1 & k = 0\\
1 & n = k\\
\displaystyle {n - 1 \choose k - 1} + {n - 1 \choose k} & \mbox{en otro caso}\\
\end{array}
\right.
$$
\codigofuente{./src/pascal_triangle.cpp}

% ---------------------------------------------------------------

% ---------------------------------------------------------------
\subsection{Permutaciones con elementos indistinguibles}

El n\'umero de permutaciones \underline{diferentes} de $n$ objetos, donde hay $n_{1}$ objetos indistinguibles de tipo 1,
$n_{2}$ objetos indistinguibles de tipo 2, ..., y $n_{k}$ objetos indistinguibles de tipo $k$, es
$$
\frac{n!}{n_{1}!n_{2}! \cdots n_{k}!}
$$
\textbf{Ejemplo:} Con las letras de la palabra \texttt{PROGRAMAR} se pueden formar $ \displaystyle \frac{9!}{2! \cdot 3!} =
30240 $ permutaciones \underline{diferentes}.
% ---------------------------------------------------------------

% ---------------------------------------------------------------
\subsection{Desordenes, desarreglos o permutaciones completas}

Un desarreglo es una permutaci\'on donde ning\'un elemento $i$ est\'a en la
posici\'on $i$-\'esima. Por ejemplo, \textit{4213} es un desarreglo de 4 elementos pero
\textit{3241} no lo es porque el 2 aparece en la posici\'on 2.

Sea $D_{n}$ el n\'umero de desarreglos de $n$ elementos, entonces:
$$ {D_{n}} = \left\{
\begin{array}{c l}
1 & n = 0\\
0 & n = 1\\
(n-1)(D_{n-1} + D_{n-2}) & n \geq 2\\
\end{array}
\right.
$$

Usando el principio de inclusi\'on-exclusi\'on, tambi\'en se puede encontrar la f\'ormula

\begin{equation*}
D_{n} = n!\left [ 1 - \frac{1}{1!} + \frac{1}{2!} - \frac{1}{3!} + \cdots + (-1)^{n}\frac{1}{n!} \right ]
= n! \sum_{i=0}^{n} \frac{(-1)^{i}}{i!}
\end{equation*}
% ---------------------------------------------------------------

% ---------------------------------------------------------------

\section{Librer\'ias de Java}

\subsection{BigInteger}

\textbf{Field-testing:}
\begin{itemize}
\item \emph{SPOJ} - 24 - Small Factorials
\end{itemize}
\codigofuente{./src/bigInteger.java}

% ---------------------------------------------------------------
% ---------------------------------------------------------------

% ---------------------------------------------------------------
% ---------------------------------------------------------------

\section{Comparaci\'on de Strings}

\subsection{KMP}
\textbf{Field-testing:}
\begin{itemize}
\item \emph{SPOJ} - 32 - A Needle in the Haystack
\end{itemize}
\codigofuente{./src/kmp.cpp}

% ---------------------------------------------------------------


% ---------------------------------------------------------------

\subsection{Distancia de Levenshtein}

\textbf{Field-testing:}
\begin{itemize}
\item \emph{SPOJ} - 6219 - Edit distance
\end{itemize}

M\'inimo n\'umero de transformaciones para convertir el String $A$ en el String $B$.
\codigofuente{./src/levenshtein.cpp}
\codigofuente{./src/levenshtein2.cpp}
% ---------------------------------------------------------------
% ---------------------------------------------------------------

% ---------------------------------------------------------------
% ---------------------------------------------------------------

\section{Grafos y \'arboles}


\subsection{Dijkstra}

\textbf{Field-testing:}
\begin{itemize}
\item \emph{SPOJ} - 15 - The Shortest Path
\end{itemize}
\codigofuente{./src/dijkstra.cpp}
% ---------------------------------------------------------------

% ---------------------------------------------------------------
\subsection{DFS}

\textbf{Field-testing:}
\begin{itemize}
\item \emph{SPOJ} - 1437 - Longest path in a tree
\end{itemize}
\codigofuente{./src/dfs.cpp}
% ---------------------------------------------------------------

% ---------------------------------------------------------------
\subsection{BFS}

\textbf{Field-testing:}
\begin{itemize}
\item \emph{SPOJ} - 38 - Labyrinth
\end{itemize}
\codigofuente{./src/bfs.cpp}
% ---------------------------------------------------------------

% ---------------------------------------------------------------
\subsection{Lowest Common Ancestor}

\codigofuente{./src/lca.cpp}
% ---------------------------------------------------------------

% ---------------------------------------------------------------
\subsection{Kruskal + Union-Find}

\codigofuente{./src/kruskal.cpp}
% ---------------------------------------------------------------

% ---------------------------------------------------------------
\subsection{Minimum Spanning Tree - Prim}

\codigofuente{./src/prim.cpp}
% ---------------------------------------------------------------

% ---------------------------------------------------------------
\subsection{M\'aximo Flujo: Ford-Fukerson}

\textbf{Field-testing:}
\begin{itemize}
\item \emph{SPOJ} - 3868 - Total Flow
\end{itemize}
\codigofuente{./src/fordFulkerson.cpp}
% ---------------------------------------------------------------

% ---------------------------------------------------------------
\subsection{Floyd - Warshall}

\codigofuente{./src/floyd.cpp}
% ---------------------------------------------------------------

% ---------------------------------------------------------------
\subsection{Sparse Table}

Range Minimum Query (RMQ). Para encontrar la posici\'on con el m\'inimo valor de un elemento entre dos \'indices espec\'ificos.
\codigofuente{./src/sparseTree.cpp}
% ---------------------------------------------------------------

% ---------------------------------------------------------------
\subsection{Segment Tree}

\codigofuente{./src/segment-tree.cpp}
% ---------------------------------------------------------------

% ---------------------------------------------------------------
\section{Checklist para WA, TLE y RE}

\begin{itemize}
  \begin{item}
    Overflow.
  \end{item}

 \begin{item}
  Requiere BigInteger.
 \end{item}

  \begin{item}
    El programa termina anticipadamente por la condici\'on en el ciclo de lectura.
    Por ejemplo, se tiene \verb_while (cin >> n >> k && n && k)_ y un caso v\'alido de entrada
    es n = 1 y k = 0.
  \end{item}

  \begin{item}
    No hay m\'as chocolatina para partir.
  \end{item}
  
  \begin{item}
    El grafo no es conexo.
  \end{item}
  \begin{item}
    Puede haber varias aristas entre el mismo par de nodos.
  \end{item}

  \begin{item}
    Las aristas pueden tener costos negativos.
  \end{item}

  \begin{item}
    El grafo tiene un s\'olo nodo.
  \end{item}

  \begin{item}
    La cadena puede ser vac\'ia.
  \end{item}

  \begin{item}
    Las l\'ineas pueden tener espacios en blanco al principio o al final (Cuidado al usar \texttt{getline} o \texttt{fgets}).
  \end{item}

  \begin{item}
    El arreglo no se limpia entre caso y caso.
  \end{item}

  \begin{item}
   Se est\'a imprimiendo una l\'inea en blanco con un espacio (\verb_printf(" \n")_ en vez de \verb_printf("\n")_ \'o \verb_puts(" ")_ en vez de \verb_puts("")_).
  \end{item}
  
 \end{itemize}
 % ---------------------------------------------------------------

% ---------------------------------------------------------------
\end{document}